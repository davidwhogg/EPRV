% var.tex : a paper about EPRV in the presence of spectral variability
% This file is part of the EPRV project.
% Copyright 2018 the authors.

% Style notes
% -----------
% - Re-define acronyms in each section? So less flippy flippy?

% To-do items
% -----------
% - figure out what experiments to do and do them?
% - Draft it.
% - Get a 6th keyword; re-consider keywords.

\documentclass[modern]{aastex62}
\usepackage{graphicx}
\usepackage{xcolor}
\usepackage[sort&compress]{natbib}
\usepackage[hang,flushmargin]{footmisc}

% units macros
\newcommand{\unit}[1]{\mathrm{#1}}
\newcommand{\km}{\unit{km}}
\newcommand{\m}{\unit{m}}
\newcommand{\s}{\unit{s}}
\newcommand{\kms}{\km\,\s^{-1}}
\newcommand{\ms}{\m\,\s^{-1}}
\newcommand{\ang}{\text{\normalfont\AA}}

% math macros
\newcommand{\dd}{\mathrm{d}}
\newcommand{\T}{^{\mathsf{T}}}

% text macros
\newcommand{\documentname}{\textsl{Article}}
\newcommand{\sectionname}{Section}
\newcommand{\todo}[1]{\textcolor{red}{#1}}  % gotta have \usepackage{xcolor} in main doc or this won't work
\newcommand{\acronym}[1]{{\small{#1}}}
\newcommand{\project}[1]{\textsl{#1}}
\newcommand{\foreign}[1]{\textsl{#1}}
\newcommand{\HARPS}{\project{\acronym{HARPS}}}
\newcommand{\HIRES}{\project{\acronym{HIRES}}}
\newcommand{\RV}{\acronym{RV}}
\newcommand{\CRLB}{\acronym{CRLB}}

\setlength{\parindent}{1.4em} % trust in Hogg
\shorttitle{precise radial velocities for variable stars}
\shortauthors{hogg and others}

\begin{document}\sloppy\sloppypar\raggedbottom\frenchspacing % trust in Hogg
\graphicspath{ {figures/} }
\DeclareGraphicsExtensions{.pdf,.eps,.png}

\title{\textbf{Precise radial-velocity measurements in the presence of stellar variability}}

\author[0000-0003-2866-9403]{David W. Hogg}
\affil{Flatiron Institute, 162 Fifth Ave, New York, NY 10010, USA}
\affil{Center for Cosmology and Particle Physics, Department of Physics, New York University, 726 Broadway, New York, NY 10003, USA}
\affil{Center for Data Science, New York University, 60 Fifth Ave, New York, NY 10011, USA}
\affil{Max-Planck-Institut f\"ur Astronomie, K\"onigstuhl 17, D-69117 Heidelberg}

\author{Others}

\begin{abstract}\noindent
% context
Spectrographs that deliver the very most precise radial-velocity (RV)
measurements of stars---for the purpose of detecting exoplanets---are
obtaining empirical root-variances of around $1\,\ms$.
This is impressive, but not as precise as information theory
permits.
One possible source of variance is stellar variability, from stellar
surface convection, sunspots, and activity.
Spectral variability is undeniably present at some level for all
convective-surface stars, and it undermines the fundamental
assumptions of essentially all extreme-precision methodologies at
present.
% aims
Here we develop extreme-precision (EPRV) estimators that
deliver RV precision without assuming that the stellar spectrum is
constant in time.
These methods are designed to work in pathological cases, where the
variability is large, so that they will perform extremely well in more
realistic situations in which the variability is small.
% methods
They are probabilistic models, built on two important assumptions:
The first is that the stellar variability is somehow compact, either
in dimensionality or in total variance, or both.
The second is that---in the limit of large data---the stellar
variability is uncorrelated with the RV signals from any planets
orbiting the star.
% results
We find that, given sufficient data, our methods converge to saturating
information-theoretic RV bounds, provided that the assumptions hold.
This methodology opens up the possibility improving all past and current
EPRV measurements, and broadening the set of stars available for EPRV study.
\end{abstract}

\keywords{
% atmospheric effects
% ---
binaries: spectroscopic
---
methods: data analysis
---
methods: statistical
---
techniques: radial velocities
---
planets and satellites: detection
}

\section*{~}
\clearpage
\section{Introduction}

Hello World!

\section{Generalities}

We have shown this and that...

We don't want to fit the spectrum with the same model with which we make the fake data.
So if we are going to fit the data with a model that is a linear superposition of eigenspectra,
we will not generate fake data as a linear superposition of eigenspectra.

We are trying to be adversarial and extreme.
Thus our methods will be stronger than they need to be.

Do we need to define different concepts of RV?
Stellar center-of-mass RV, and stellar surface RV?
Do we want to have a pure RV component to the stellar variability in our fake data?

We can turn on and off regularization of the stellar com RV to be generated by a Kepler process,
or a mixture of Keplers?

\section{Fake data}

We have a baseline fake spectrum, shown in ??.

We permit every line center and depth to vary on a one-dimensional latent $z$.

We permit the line shape to vary on a one-dimensional latent $w$.

We add in a surface RV according to one-dimensional latent $u$?

Note that this basis is not in any sense a linear basis!
It would be close to linear if the line shifts were small relative to the line widths,
but they aren't.
That said, it is still low dimensional even in the linear sense.
That will be important for what follows.

\section{Method: wobble}

Hello World!

\section{Experiments and results}

Hello World!

\section{Discussion}

Hello World!

\acknowledgements
It is a pleasure to thank
  Megan Bedell (Flatiron),
  Benjamin Pope (NYU),
and the participants in the weekly Stars Group Meeting at the Flatiron
Institute for valuable contributions to this project.

\todo{...add Hogg grant numbers}

\todo{...add facilities tags}

\todo{...add software tags}

\bibliographystyle{apj}
\bibliography{}%general,myref,inprep}

\end{document}
