\documentclass[modern]{aastex61}
\usepackage{graphicx}
\usepackage{xcolor}
\usepackage[sort&compress]{natbib}
\usepackage[hang,flushmargin]{footmisc}

\newcommand{\unit}[1]{\mathrm{#1}}
\newcommand{\km}{\unit{km}}
\newcommand{\m}{\unit{m}}
\newcommand{\s}{\unit{s}}
\newcommand{\kms}{\km\,\s^{-1}}
\newcommand{\ms}{\m\,\s^{-1}}
\newcommand\todo[1]{\textcolor{red}{#1}}  % gotta have \usepackage{xcolor} in main doc or this won't work
\newcommand{\acronym}[1]{{\small{#1}}}
\newcommand{\project}[1]{\textsl{#1}}
\newcommand{\HARPS}{\project{\acronym{HARPS}}}
\newcommand{\RV}{\acronym{RV}}
\newcommand{\CRLB}{\acronym{CRLB}}

\begin{document}
\graphicspath{ {figures/} }
\DeclareGraphicsExtensions{.pdf,.eps,.png}

\title{Achieving maximum possible precision on stellar radial-velocity measurements in multi-epoch spectroscopy}

\author{Megan Bedell}
\affil{Center for Computational Astrophysics, Flatiron Institute, 162 Fifth Ave, New York, NY 10010, USA}
\affiliation{University of Chicago, NEED DETAILS HERE}

\author[0000-0003-2866-9403]{David W. Hogg}
\affil{Center for Cosmology and Particle Physics, Department of Physics, New York University, 726 Broadway, New York, NY 10003, USA}
\affil{Center for Data Science, New York University, 60 Fifth Ave, New York, NY 10011, USA}
\affil{Max-Planck-Institut f\"ur Astronomie, K\"onigstuhl 17, D-69117 Heidelberg}

\begin{abstract}
% context
Extremely precise stellar radial velocity (\RV) measurements are a foundational tool in the field of exoplanets, with hundreds of planets discovered by the Doppler method to date and about a dozen high-resolution spectrographs around the world dedicated to the search.
Despite the importance of these observations, the detailed computation methods used to extract $\ms$-level \RV s from stellar spectra remain largely unstandardized and unpublished.
% aims
In this work, we look at sensible approaches to extracting \RV s from a data set of  one-dimensional, calibrated, extracted spectra and compare their efficiencies by testing on a segment of realistic model spectrum, with the goal of identifying methods that have the possibility of saturating the information-theoretic Cram\'er--Rao bound (\CRLB) on measurement precision.
% methods
We consider cross-correlations (and, equivalently, maximum-likelihood optimizations) of the data with generalized (weighted) binary masks, synthetic spectra (from, say, stellar models), and data-driven templates constructed from the data themselves.
We also consider the possibility that there are low-amplitude, unknown telluric absorptions (micro-tellurics) affecting the spectrum, and continuum-normalization problems, and measure improvements that flow from modeling those also in a data-driven way.
% results
We find that of all methods, only cross-correlation with an iteratively built data-driven template comes close to saturating the \CRLB.
We find that if there are micro-tellurics, they can be captured by a data-driven model.
We predict that simultaneously fitting data-driven models of the stellar spectrum and telluric contributions at all observational epochs will improve the \RV\ precision of \HARPS\ to substantially better than $1\,\ms$.
\end{abstract}

\section{Introduction}

Accurate spectroscopic measurements of radial velocities (\RV s) have long been a critical tool across many subfields of astronomy, from galactic dynamics to cosmic expansion. In recent decades, however, a new focus has emerged on \textit{extremely precise} \RV\ measurements of stars as a means of detecting exoplanets. For this application, measuring the absolute velocity of the star is much less important than resolving its changes through time. With sub-$\ms$-level precision required for the detection of Earth-like planets, vast amounts of effort have been poured into the engineering challenges of constructing a sufficiently stable spectrograph (or, failing that, being able to track the drift of a spectrograph over time) \todo{(cite some review papers)}. At the same time, great strides have been made towards disentangling true Doppler shifts from imitative signals like the spectral signatures of stellar pulsations and magnetic activity features \todo{(cite some review papers)}.  In this work, we investigate another fundamentally important but less commonly discussed aspect of the quest for maximum-precision \RV s: the method of analysis used to extract the \RV\ from a stellar spectrum.

While the general idea of measuring a Doppler shift by the relative motions of stellar absorption lines is fairly straightforward, there are considerable subtleties to implementing this, from the choice of template 

We begin in Section \ref{s:methods} with a summary of the practices used by current \RV\ pipelines, including cross-correlation and template fitting. We then implement these techniques to extract \RV s from a 5-\r{A} section of simulated spectrum and compare their performance in Section \ref{s:fakedata}. 

[summary of the ways people derive precise \RV s from spectra; citations to binary mask papers, papers that compare template matching to binary masks]

\section{Overview of Methods}
\label{s:methods}

cross-correlation vs. max likelihood

for either of these, you could use:
\begin{itemize}
\item ``binary'' mask
\item synthetic spectrum (or template observation?)
\item data-driven template spectrum
\end{itemize}

information theory based discussion of these: compute the CRLB for velocity

\begin{figure}
\centering
% \includegraphics[width=\columnwidth]{binarymask}
\caption{Illustration of a weighted binary mask (blue windows) plotted over a small segment of the solar spectrum (black). The vertical length of each mask window corresponds to the weight given to the line when performing a cross-correlation between mask and spectrum.}
\label{fig:binarymask}
\end{figure}


\section{Testing with Toy-Model Spectra}

The basic model that we use to generate fake spectra is extremely simple, assuming a perfectly normalized continuum; isolated, perfectly Gaussian spectral lines; and white noise only. We simulated a 5-$\AA$ region with random noise corresponding to an SNR of 100, a typical value for a single \RV\ spectrum. Lines were inserted as Gaussians with arbitrarily-prescribed centers and depths (or equivalent widths). Every line was convolved with the same line spread function, which was taken to be a Gaussian with $\sigma$ = 0.05 $\AA$. %The number of lines used varies in the results discussed below.

We repeated this synthesis to make a set of 512 spectra, each with a random \RV\ between $-30$ and $30\,\kms$ (the approximate amplitude of the yearly \RV\ shifts induced by barycentric motion). The \RV s can be applied as a straightforward Doppler shift in the centers of each line. \todo{(add equation?)} At this point, we are able to run a cross-correlation or a maximum-likelihood analysis on the set of spectra and examine the deviations in the recovered \RV s from the input \RV s.

\todo{remember to discuss quadratic max procedure.. probably in previous section?}

\subsection{Single-Line Results}

\subsection{Multi-Line Results}

\section{Dependence of RV Precision on Template Quality}

how well does the mask/template approximate the real data? try fudging the relative line depths and locations.

\section{Correlated Noise Sources and Their Consequences}

additional noise considerations:
\begin{itemize}
\item (micro-) telluric features
\item subtle continuum trends
\item cosmic rays?
\end{itemize}

\section{Application to Real Spectra}

how do our results compare to the reality of the situation? (will this tell us anything other than ``real spectra are more complicated and our template is silly''?)

\section{Discussion}

\section{Conclusions}

\acknowledgements
MPIA hospitality

\bibliographystyle{apj}
\bibliography{}%general,myref,inprep}

\end{document}
