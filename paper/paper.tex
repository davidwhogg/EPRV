\documentclass[modern]{aastex61}
\usepackage{graphicx}
\usepackage{xcolor}
\usepackage[sort&compress]{natbib}
\usepackage[hang,flushmargin]{footmisc}

\newcommand{\kms}{km\,s$^{-1}$}
\newcommand{\ms}{m\,s$^{-1}$}
\newcommand\todo[1]{\textcolor{red}{#1}}  % gotta have \usepackage{xcolor} in main doc or this won't work



\begin{document}
\graphicspath{ {figures/} }
\DeclareGraphicsExtensions{.pdf,.eps,.png}

\title{Extracting Maximum-Precision Radial Velocities from High-Resolution Spectra}

\author{Megan Bedell}
\affiliation{Flatiron Institute}
\affiliation{University of Chicago}

\author{David W. Hogg}
\affiliation{Flatiron Institute}
\affiliation{CCPP, NYU}
\affiliation{CDS, NYU}
\affiliation{MPIA}

\begin{abstract}
Precise stellar radial velocity (RV) measurements are a foundational tool in the field of exoplanets, with hundreds of planets discovered by the Doppler method to date and about a dozen high-resolution spectrographs around the world dedicated to the search. Despite the importance of these observations, the detailed computation methods used to extract \ms-level RVs from stellar spectra remain largely unstandardized and unpublished. In this work, we provide a brief overview of possible approaches to extracting RVs and compare their efficiencies by testing on a segment of realistic model spectrum. We find that cross-correlation with an iteratively built data-driven template outperforms traditional cross-correlation with a quasi-binary mask by an order of magnitude and saturates the Cram\'{e}r-Rao bound under white-noise conditions. We additionally simulate the effects of unresolved micro-telluric lines and quantify their influence on the achievable RV precision. We predict that simultaneously fitting data-driven models of the stellar spectrum and telluric contributions at all observational epochs will improve the RV precision of HARPS to substantially better than 1 \ms.
\end{abstract}

\section{Introduction}

Accurate spectroscopic measurements of radial velocities (RVs) have long been a critical tool across many subfields of astronomy, from galactic dynamics to cosmic expansion. In recent decades, however, a new focus has emerged on \textit{extremely precise} RV measurements of stars as a means of detecting exoplanets. For this application, measuring the absolute velocity of the star is much less important than resolving its changes through time. With sub-\ms-level precision required for the detection of Earth-like planets, vast amounts of effort have been poured into the engineering challenges of constructing a sufficiently stable spectrograph (or, failing that, being able to track the drift of a spectrograph over time) \todo{(cite some review papers)}. At the same time, great strides have been made towards disentangling true Doppler shifts from imitative signals like the spectral signatures of stellar pulsations and magnetic activity features \todo{(cite some review papers)}.  In this work, we investigate another fundamentally important but less commonly discussed aspect of the quest for maximum-precision RVs: the method of analysis used to extract the RV from a stellar spectrum.

While the general idea of measuring a Doppler shift by the relative motions of stellar absorption lines is fairly straightforward, there are considerable subtleties to implementing this, from the choice of template 

We begin in Section \ref{s:methods} with a summary of the practices used by current RV pipelines, including cross-correlation and template fitting. We then implement these techniques to extract RVs from a 5-\r{A} section of simulated spectrum and compare their performance in Section \ref{s:fakedata}. 

[summary of the ways people derive precise RVs from spectra; citations to binary mask papers, papers that compare template matching to binary masks]

\section{Overview of Methods}
\label{s:methods}

cross-correlation vs. max likelihood

for either of these, you could use:
\begin{itemize}
\item ``binary'' mask
\item synthetic spectrum (or template observation?)
\item data-driven template spectrum
\end{itemize}

information theory based discussion of these: compute the CRLB for velocity. Mention how uncertainties are typically computed (Butler 1996 and Bouchy 2001 formulae).

\begin{figure}
\centering
\includegraphics[width=\columnwidth]{binarymask}
\caption{Illustration of a weighted binary mask (blue windows) plotted over a small segment of the solar spectrum (black). The vertical length of each mask window corresponds to the weight given to the line when performing a cross-correlation between mask and spectrum.}
\label{fig:binarymask}
\end{figure}


\section{Testing with Toy-Model Spectra}

The basic model that we use to generate fake spectra is extremely simple, assuming a perfectly normalized continuum; isolated, perfectly Gaussian spectral lines; and white noise only. We simulated a 5-$\AA$ region with random noise corresponding to an SNR of 100, a typical value for a single RV spectrum. Lines were inserted as Gaussians with arbitrarily-prescribed centers and depths. Every line was convolved with the same line spread function, which was taken to be a Gaussian with $\sigma$ = 0.05 $\AA$.

We repeated this synthesis to make a set of 512 spectra, each with a random RV between -30 \kms and 30 \kms (the approximate amplitude of the yearly RV shifts induced by barycentric motion). The RVs can be applied as a straightforward Doppler shift on the central wavelength of each line. \todo{(add equation?)} At this point, we are able to run a cross-correlation or a maximum-likelihood analysis on the set of spectra and examine the deviations in the recovered RVs from the input RVs.

\todo{remember to discuss quadratic max procedure.. probably in previous section?}

\subsection{Single-Line Results}

We began with the simplest test case possible: a single, strong absorption feature.

What is the CRLB?

Discuss scaling of the CRLB with line FWHM, constrast/depth, etc. According to Lovis \& Fischer (2010) and Figueira (2017), $\sigma_{RV}$ should go as $\frac{\sqrt{FWHM}}{SNR \times C}$, where $C$ is the line contrast, or the depth of the line as a fraction of the continuum level \todo{(check on this definition of contrast)}. Do we see this?

\subsection{Multi-Line Results}

Our next test involved a more realistic test spectrum. We measured the centers and depths of \todo{N} lines in the Solar spectrum between {x - y} $\AA$ and used these measurements to generate a synthetic spectrum that closely resembles a typical observation of a Sun-like star.

What is the CRLB for this case? Binary mask cross-correlation is very far from saturating it. Cross-correlation or max-likelihood fitting with a perfect spectral template works much better. Talk a little bit about why.

\section{Dependence of RV Precision on Template Choice}

how well does the mask/template approximate the real data? try fudging the relative line depths and locations in both the binary mask and the template.

is this where we bring in the data-driven template and demonstrate how well it performs?

\section{Correlated Noise Sources and Their Consequences}

additional noise considerations:
\begin{itemize}
\item (micro-) telluric features
\item subtle continuum trends
\item cosmic rays?
\end{itemize}

\section{Application to Real Spectra}

how do our results compare to the reality of the situation? (will this tell us anything other than ``real spectra are more complicated and our template is silly''?)

%\section{Discussion}

\section{Conclusions}

\acknowledgements
MPIA hospitality

\bibliographystyle{apj}
\bibliography{}%general,myref,inprep}

\end{document}