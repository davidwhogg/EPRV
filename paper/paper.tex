\documentclass[twocolumn]{aastex61}
\usepackage{graphicx}

\usepackage[sort&compress]{natbib}
\usepackage[hang,flushmargin]{footmisc}

\begin{document}
\graphicspath{ {figures/} }
\DeclareGraphicsExtensions{.pdf,.eps,.png}

\title{How to Extract Maximum-Precision Radial Velocities from High-Resolution Spectra}

\author{David W. Hogg}
\affiliation{a lot of places}

\author{Megan Bedell}
\affiliation{Flatiron Institute?}

\begin{abstract}

\end{abstract}

\section{Introduction}

summary of the ways people derive precise RVs from spectra; citations to binary mask papers, papers that compare template matching to binary masks

\section{Overview of Methods}

cross-correlation vs. max likelihood

for either of these, you could use:
\begin{itemize}
\item ``binary'' mask
\item synthetic spectrum
\item data-driven synthetic spectrum
\end{itemize}

information theory based discussion of these: compute the CRLB for velocity

\section{Generating Test Spectra}

how the model works in general, Doppler shifting, parameterization of white noise. number of lines.

additional noise considerations:
\begin{itemize}
\item (micro-) telluric features
\item subtle continuum trends
\item cosmic rays?
\end{itemize}

another important consideration is how well the mask/template fits the data. summarize masks and templates tried (do they know the correct relative line depths and locations?).

\section{Single-Line Results}

\section{Multi-Line Results}

\section{Discussion}

\section{Conclusions}

\acknowledgements

\bibliographystyle{apj}
\bibliography{}%general,myref,inprep}

\end{document}